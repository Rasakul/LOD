\subsection{Methodology}
In this study the data were acquired by a coordinated set of semi-structured interviews. As mentioned the stakeholders were classified into three groups (\textit{administrative staff}, \textit{students} and \textit{researchers}) and therefore three different versions of the questionnaire but with joint parts for statistically evaluation were worked out. For each version exists an according paper, in this work only the category "`researcher"' will be described.

\subsubsection{Design of questionnaire}
The main purpose of the interviews were the collecting of the stakeholders thoughts, needs and knowledge, so the method of of a semi-structured interview was chosen. A \textit{fully structured interview} would not be adequate because of it's strict character allowing only predefined answers and a \textit{unstructured interview} would be to difficult to analyze.

After choosing the method, the questionnaire was defined. To allow a general, generic shared analyze of the interviewees the team decided to mix open questions from the semi-structured model with closed questions with fixed, predefined questions. The result had four parts:

\begin{enumerate}
	\item General question about the interviewee for classification, about his/her work
	\item General question about the interviewee's knowledge in general technical and LOD context. This part is the part for statistical evaluation.
	\item Explanation of LD, followed by a specific set of questions targeting the thoughts and opinions of the interviewee about presented use cases and example application. Motivation of this part is to introduce the interviewee to LOD if it is an unknown topic and let him/her start to think about LOD to prepare the next part
	\item Wide open Questions to explore and find use cases and existing data sources for LOD application at the university.
\end{enumerate}

The examples from part 3 were LD in libraries (see~\ref{ld-libraries}) and an obvious source of research related data: the publication database.

\subsubsection{Description of interviewed people}
As mentioned the interviewees of this study were chosen according to the category "`\textit{Research}"', so the interview partner were active researcher in various fields. Altogether four interviews were done. Because of the technical character of LOD the chosen people are all technically experienced so they are able to imagine use cases at the university. In future work there is a need of more less experienced researchers to understand their thoughts.

\subsubsection{Data Validity and Quality}
To ensure both a continuous conversation flow and a high quality recording of the spoken words, the interviews were held in teams, one speaker and one writer making notes. Additional all interviews were audio recorded.
As result the data are available as interview notes and audio records.