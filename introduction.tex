\section{Introduction}
While the pressure on governments and public organizations to release \textit{Open~Data~(OD)} has significantly grown with the spread of information systems, there has been also a need for \textit{linking} these data from various sources to understand the information as a whole.

Open Data includes non-privacy-restricted and non-confidential data. Therefore any restrictions in distribution are prohibited and data is funded only by public money.~\citet{article:janssen2012benefits} The application domain for Open Data providers is not restricted by its nature in any way and ranges from traffic, weather, statictics to budgeting in the public sector. Just the publication of Open Data seems not enough but in addition the implementation of a feedback loop result in \textit{Open Government}. This has the advantage of a constant adaption to the citizen's needs instead of just visualizing former closed data. 

Despite its wide adoption of Open Data it does restrict the published Data format in any way, thus complicating the integration of heterogeneous data sets. The World Wide Web has proven great success spreading knowledge of various data sources all over the world. The building block of the Web are documents and links connecting them to form a global information space. This can be seen as the key success factor in its nearly unconstrained growth~\cite{report:jacobs-i-2004--a}. 
Following these principles of publishing and connecting data on the Web is known as \textit{Linked~Data~(LD)}. More technically, it refers to machine-readable data which is linked to external data sets and can in turn be linked from other data sets. 

\citet{artivle:bernerslee-t-2006-1} developed a five star rating scheme for classifying \textit{Linked Open Data~(LOD)}, which combines Linked Data and Open Data. The scheme ranges from one star describing Open Data only to five stars describing Open Data in a machine-readable format using open standards with links to other data sets.

Although Linked~Open~Data offer universities new opportunities for providing unprecedented insight into its core activities and ease application development, a major \textbf{problem} is that \textbf{Linked~Open~Data has not been widely adopted by universities yet}. Even tough there are a few examples~\cite{url:linked-universities-members} of publishing university related data as Linked~Open~Data, there has been little knowledge of using Linked~Open~Data for publishing university related information. 

The remainder of this section states the addressed research question of this paper, describes the contributions in the course of investigating the research questions and gives an overview of the structure of this paper.

\subsection{Research Questions}
The fundamental research questions underlying this paper is:
\begin{displayquote}
\textit{How can Linked~Open~Data help to improve processes in university context and how can it be successfully applied?}
\end{displayquote}
More concrete this paper concentrates on the following more concrete research questions:
\paragraph{Q1: What are best practices regarding the applicability of Linked Open Data in university settings?}
As of now, there are no established best practices for the use of Linked Open Data due to its little adoption in university contexts. For this very reason it is crucial to identify strengths and limitation from previous experiences~\cite{url:linked-universities-members} of using Linked~Open~Data as core technology. 
\paragraph{Q2: What are major benefits and barriers for each stakeholder and what are useful use cases?}
We identified three different stakeholders in university context \textit{Students}, \textit{Researchers} and \textit{Administration staff}. Since the success of any new technology highly depends on the acceptance of the stakeholders, the needs of each of the target groups needs to be examined. Furthermore, use cases are important to showcase profits and shortcomings to a non-technical audience. 
\paragraph{Q3: What are major challenges for the implementation of a Linked Open Data~solution?}
As the implementation of a Linked Open Data solution is a time consuming task, the knowledge of probable challenges from the technical perspective as well as from the management perspective is a key factor for the successful adoption. 
\paragraph{Q4: How would a prototypical implementation of Linked~Open~Data look like?}
Among the various existing data sets available it needs to be investigated if a (semi-) automatic transformation is feasible or is the manual data provision enough. In addition, from an implementers perspective of view, critical factors regarding the storage and retrieval of Linked~Open~Data need to be identified. 

\subsection{Methodology}
Finding an answer for the research questions above has lead to the following three methodologies:
\paragraph{A coordinated set of semi-structured interviews}
To answer research questions~(RQ) two to four, we interviewed a selected set of stakeholders representing \textit{Students}, \textit{Researchers} and \textit{Administration staff} respectively. Semi-structured interviews were selected as the means of data collection because they are well suited for exploring the impressions and interests of the interviewees as in a discussion while still following a defined structure. 
\paragraph{Litrature Review}
Undertaking a litrature review to justify scientific contributions and making sound conclusions is an established practice in any scientific community. Since our scientific work targeted in particular to the Semantic Web community, we made some pre-assumptions of a basic understanding of the technologies and concepts regarding Linked Data. More specifically, the concept of an ontology and example knowledge descriptions languages describing these will not be covered in this paper. 
\paragraph{Conceptual System Design}
The development of applications based on Linked Open Data requires a methodology which describes a common understanding of the overall system infrastructure. Therefore we designed a conceptual model of a prototypical implementation of a Linked Open Data solution. 

\subsection{Contributions}
The work in this paper mainly contributes to different aspects wich need to be considered when designing and implementing a Linked Open Data application.
More precisely, our contributions can be categorized into the following four areas:
\paragraph{1. Identifying best practices for Linked Open Data in university context.}
Due to the crowing complexity and the large amount of data information systems need to process, there has been the need to efficiently handle Linked Data as well. We gave a brief overview of the already published research work regarding Linked Open Data in university context. In particular, we compared the profits and shortcomings in existing Linked Open Data solutions. 
\paragraph{2. Finding benefits/barriers with additional use cases for stakeholders.}
As with every software project the very first phase of the Software Development Lifecycle~(SDLC) is the \textit{Evaluation of the Requirements}. As a Linked Open Data solution has additional requirements to the structure of the data and due to its open nature, we investigated if the overhead compared to an established technology (e.g. a database based solution) is worth the effort. A set of selected participants from the areas Research, Student~Affairs and Administration are interviewed at the University of Technology in Vienna and their benefits/barriers are compared. Additionally, we proposed several use cases emphasising their point of view. 
\paragraph{3. Discovering possible obstacles for implementers of a Linked Open Data Solution.}
As the application domain for a Linked Data is limited to the university context, our work includes a defined set of Linked Open Data applications which were merged together from the conducted interviews. That use cases showcased probable shortcomings which might arise before, during or after the implementation.
\paragraph{4. Sketching a prototypical implementation of a Linked Open Data Solution.}
In consideration of the above mentioned obstacles of a possible Linked Open Data Solution, we gave an outline of a prototypical implementation. It begins by covering the whole process of data provision and ends by applications made for end users. 
\subsection{Structure of this Paper}
\%\%\%\%tbd\%\%\%\%