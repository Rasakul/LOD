\subsubsection{General statistically evaluation of the interviewees background}\label{stat_general}

\begin{figure}[htbp]
\centering
\label{Fi:lod-exp}
	\begin{tikzpicture}
		\begin{axis}
			[title=Level of ICT-Expertise,
			ybar,% Balken
			% urspr�ngliche y-Werte unterhalb der Balken:
			nodes near coords,nodes near coords align=above,point meta=rawy,
			axis x line=bottom, axis y line=left,% Achsen nur unten und links,
			%xlabel=blub,ylabel=bla, % Beschriftung der Achsen
			ymin=0,% minimaler y-Wert ist 0
			xtick=data,% xticks nur an Stellen mit Daten
			enlargelimits=auto,% Vergr��ern der R�nder des Diagramms
			% Ausgabe der x Werte ohne Tausendermarkierung@
			x tick label style={/pgf/number format/1000 sep=},
			legend pos=north west]
			\addplot table[x=Question, y=Number] {data/data_general_researcher_1.csv};
			\addplot table[x=Question, y=Number] {data/data_general_students_1.csv};
			\addplot table[x=Question, y=Number] {data/data_general_admin_1.csv};
			\legend{Researcher, Students, Administration}
		\end{axis}
	\end{tikzpicture}
	\begin{tikzpicture}
		\begin{axis}		
			[title=Level of LOD-Expertise,
			ybar,% Balken
			% urspr�ngliche y-Werte unterhalb der Balken:
			nodes near coords,nodes near coords align=above,point meta=rawy,
			axis x line=bottom, axis y line=left,% Achsen nur unten und links,
			%xlabel=blub,ylabel=bla, % Beschriftung der Achsen
			ymin=0,% minimaler y-Wert ist 0
			xtick=data,% xticks nur an Stellen mit Daten
			enlargelimits=auto,% Vergr��ern der R�nder des Diagramms
			% Ausgabe der x Werte ohne Tausendermarkierung@
			x tick label style={/pgf/number format/1000 sep=},
			legend pos=north east]
			\addplot table[x=Question, y=Number] {data/data_general_researcher_2.csv};
			\addplot table[x=Question, y=Number] {data/data_general_students_2.csv};
			\addplot table[x=Question, y=Number] {data/data_general_admin_2.csv};
			\legend{Researcher, Students, Administration}
		\end{axis}
	\end{tikzpicture}
	\caption[Level of Experience]{Level of Experience}
\end{figure}

The first part of the interview aimed to categorize the interview partners and to understand their background. They had to estimate their level of expertise in the field of Information \& Communication Technologies and in the field of Linked Open Data in a formal way and describe their daily work and responsibilities at the university. This part was identical in all three interview series (administration, students and researcher), so the result can be compared. The levels of expertise can be seen in figure~\ref{Fi:lod-exp} and the according rating scales in table~\ref{table:rating-scales}. For a detailed description of the administration and student interviews see the corresponding reports~\citet{article:gamerith_publishing_2016} (administration data) and ~\citet{article:haller_publishing_2016}(data concerning students).

Tough all four research-interviewees had a high expertise in ICTs, their expertise in LOD are mainly unincisive, although everyone already knew the concept of LOD. This can easily explained by their research fields (see table~\ref{table:interviewee-background}): everyone works in a technical context. 

This could be seen as advantage as well as disadvantage. On the one hand the interviewed persons needed lesser effort to understand the benefits of LOD and could easier imagine further use cases and possible data sets for LOD. But on the other hand, their perspectives were in some way restricted by their profession - a more divergent angle of view may be interesting for further studies to explore more ``non-technical'' use cases. But for an initial study like this one it may be enough.

	
\begin{figure}[htbp]
  \centering
	\label{table:rating-scales}
	\begin{tabular}{ c | l  | p{10cm} }
		\textbf{Value} & \textbf{ICT} & \textbf{LOD}\\			\hline
		1 & Fundamental & I never heard of Linked Open Data.\\ \hline		
		2 & Novice & I heard of Linked Open Data, but never used it.\\		\hline
		3 & Intermediate & I used Linked Open Data in a not intense way. E.g. as part of a workshop or home project.\\		\hline
		4 & Advanced & I used Linked Open Data in a practical project.\\		\hline
		5 & Expert & I used Linked Open Data in several practical projects and consider myself an expert in Linked Open Data.\\
	\end{tabular}
	\caption[Rating scales]{Rating scales for the level of expertise in the field of ICT and LOD.}
\end{figure}

\begin{figure}[htbp]
  \centering
	\label{table:interviewee-background}
	\begin{tabular}{ c | l | c }
		\textbf{ID} & \textbf{Assignment} & \textbf{Date of interview} \\ \hline
		A & research \& teaching in HCI & 15.12.2015\\
		B & research in information retrieval & 04.12.2015\\
		C & research transfer & 23.11.2015\\
		D & research in audio and video analysis & 24.11.2016\\
	\end{tabular}
	\caption[Interviewees background]{Background information of the interviewees and the corresponding interviews.}
\end{figure}