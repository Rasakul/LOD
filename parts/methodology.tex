\subsection{Methodology}\label{methodology}
In this study the data were acquired by a coordinated set of semi-structured interviews. As mentioned the stakeholders were classified into three groups (\textit{administrative staff}, \textit{students} and \textit{researchers}) and therefore three different versions of the questionnaire but with joint parts for statistically evaluation were worked out. For each version exists an according paper, in this work only the category ``researcher'' will be described.

\subsubsection{Design of questionnaire}
The main purpose of the interviews were the collecting of the stakeholders thoughts, needs and knowledge, so the method of of a \textit{semi-structured} interview was chosen. A \textit{fully structured interview} would not be adequate because of its strict character allowing only predefined answers and a \textit{unstructured interview} would be top difficult to analyze. In contrast a \textit{semi-structured} interview let researchers have their freedom to decide, which question to ask next in dependence on the development of the conversation - this allows the freedom to express and evaluate new ideas without being being constrained by a
fixed order of planned questions.

After choosing the method the questionnaire was defined to use as guidline during the interviews. To allow a general, generic shared analyze of the interviewees the team decided to mix open questions from the semi-structured model with closed questions with fixed, predefined questions. The result had four parts:

\begin{enumerate}
	\item General question about the interviewee for classification, about his/her work
	\item General question about the interviewee's knowledge in general technical and LOD context.
	\item Explanation of LD, followed by a specific set of questions targeting the thoughts and opinions of the interviewee about presented use cases and example application. Motivation of this part is to introduce the interviewee to LOD if it is an unknown topic and let him/her start to think about LOD to prepare the next part
	\item Wide open Questions to explore and find use cases and existing data sources for LOD application at the university.
\end{enumerate}

Part 1 and 2 are used for a general statistical evaluation (see section~\ref{stat_general}). In Part 3 were two concrete use cases proposed: publishing (meta-)data from the library similar to the LD4D project (see~\ref{ld-libraries}) and publishing the publication data of the university from the publication database, according to the similar ``Open Research Online''~\footnote{\url{http://data.open.ac.uk/page/context/oro}} endpoint from the Open University.

The list above represents indeed the order of the parts addressed during the interview, but due the semi-structured design the questionnaire allowed variance in part 3 and 4. E.g. could the length and depth of the LOD introduction be adopted to the level of expertise of the interview partner.

\subsubsection{Data Validity and Quality}
To ensure both a continuous conversation flow and a high quality recording of the spoken words, the interviews were conducted in teams of two researchers. One researcher, who was the moderator and one, who had the task to collect data by writing valuable contributions onto a logging sheet. This approach was chosen, because the attempt to moderate a interview and to simultaneously log it may lead to the lose of valuable contributions or a frequently interrupted conversation. Further to avoid too detailed notes all interviews were additionally audio recorded so no information was lost and the second researcher could easier follow the interview and make more valuable notes.
As result the data are available as interview notes and audio records and further the evaluation could be done without replay the whole interview or transcribe it.

\subsubsection{Description of interviewed people}
As mentioned the interviewees of this study were chosen according to the category ``\textit{Research}'', so the interview partner were active researcher in various fields. Altogether four interviews were done. Because of the technical character of LOD the chosen people are all technically experienced so they are able to imagine use cases at the university - for a detailed describtion of the interviewees background see section~\ref{stat_general}.