\begin{figure}[htbp]
\centering
	\begin{tikzpicture}
		\begin{axis}		
			[title=Level of IT-Experience,
			ybar stacked,% Balken gestapelt
			% urspr�ngliche y-Werte unterhalb der Balken:
			nodes near coords,nodes near coords align=above,point meta=rawy,
			axis x line=bottom, axis y line=left,% Achsen nur unten und links,
			%xlabel=blub,ylabel=bla, % Beschriftung der Achsen
			ymin=0,% minimaler y-Wert ist 0
			xtick=data,% xticks nur an Stellen mit Daten
			enlargelimits=auto,% Vergr��ern der R�nder des Diagramms
			% Ausgabe der x Werte ohne Tausendermarkierung@
			x tick label style={/pgf/number format/1000 sep=}]
			\addplot table[x=Question, y=Number] {data_general_2.csv};
		\end{axis}
	\end{tikzpicture}
	\begin{tikzpicture}
		\begin{axis}		
			[title=Level of LOD-Experience,
			ybar stacked,% Balken gestapelt
			% urspr�ngliche y-Werte unterhalb der Balken:
			nodes near coords,nodes near coords align=above,point meta=rawy,
			axis x line=bottom, axis y line=left,% Achsen nur unten und links,
			%xlabel=blub,ylabel=bla, % Beschriftung der Achsen
			ymin=0,% minimaler y-Wert ist 0
			xtick=data,% xticks nur an Stellen mit Daten
			enlargelimits=auto,% Vergr��ern der R�nder des Diagramms
			% Ausgabe der x Werte ohne Tausendermarkierung@
			x tick label style={/pgf/number format/1000 sep=}]
			\addplot table[x=Question, y=Number] {data_general_3.csv};
		\end{axis}
	\end{tikzpicture}
	\caption[Level of Experience]{Level of Experience}\label{Fi:lod-exp}
\end{figure}